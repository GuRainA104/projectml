\documentclass[UTF8]{ctexart}
\usepackage{geometry}
\usepackage{listings}
\usepackage{xcolor}
\usepackage{graphicx}
\usepackage{float}
\usepackage{multirow}
\usepackage{array}
\usepackage{longtable}
\usepackage{hyperref}
\usepackage{ctex}
\usepackage{graphicx}
\usepackage{amsmath}
\usepackage{appendix}

\geometry{a4paper,scale=0.8}
\lstset{
    numbers=left, %设置行号位置
    numberstyle=\tiny, %设置行号大小
    keywordstyle=\color{blue}, %设置关键字颜色
    commentstyle=\color[cmyk]{1,0,1,0}, %设置注释颜色
    frame=single, %设置边框格式
    escapeinside=``, %逃逸字符(1左面的键),用于显示中文
    breaklines, %自动折行
    extendedchars=false, %解决代码跨页时,章节标题,页眉等汉字不显示的问题
    xleftmargin=2em,xrightmargin=2em, aboveskip=1em, %设置边距
    tabsize=4, %设置tab空格数
    showspaces=false %不显示空格
    }
\hypersetup{
    colorlinks=true,
    linkcolor=black,
    citecolor=black
}
\title{机器学习\\小组作业 集成电路设计}
\author{张金钊 \ 徐子飞 \ 史习恩}
\date{\today}


\begin{document}
\maketitle
\thispagestyle{empty}
 
\tableofcontents
\section{任务1: 逻辑综合评价预测}
\subsection{提取AIG文件以及评价计算}
这里我们按照实例所示代码进行尝试,定义logicSynthesis()函数用于读取AIG文件,
\begin{lstlisting}[language=python]
    import os
    import re
    import abc_py
    import pickle
    
    def logicSynthesis(state):
        # 1. 从状态字符串中提取电路名称和合成操作序列
        circuitName, actions = state.split('_')
        
        # 2. 构建电路文件路径和库文件路径
        circuitPath = f'/home/genshinmaster/project/InitialAIG/train/{circuitName}.aig'
        libFile = '/home/genshinmaster/project/lib/7nm/7nm.lib'
        logFile = f'{circuitName}.log'
        nextState=f'/home/genshinmaster/yosys/AIGFiles/{state}.aig' # 当前AIG文件
        
        # 3. 定义合成操作与Yosys命令的映射
        synthesisOpToPosDic = {
            0: "refactor",
            1: "refactor -z",
            2: "rewrite",
            3: "rewrite -z",
            4: "resub",
            5: "resub -z",
            6: "balance"
        }
        
      
       # 4. 构建Yosys命令序列
        action_cmd = ''
        for digit in actions:
            if digit.isdigit():
                operation = int(digit)
                if operation in synthesisOpToPosDic:
                    action_cmd += f"{synthesisOpToPosDic[operation]}; "
                else:
                    print(f"Warning: Operation {operation} not found in the dictionary.")
        
        
        # 5. 执行Yosys命令
        abcRunCmd = f"./yosys-abc -c \"read {circuitPath}; {action_cmd}; read_lib {libFile}; write {nextState}; print_stats\" > {logFile}"
        
        os.system(abcRunCmd)
         
    
    def evaluate_aig(aig_file, lib_file,log_file):
        """
        评估 AIG 文件的优劣
        
        参数:
        aig_file (str): AIG 文件路径
        lib_file (str): 库文件路径
        
        返回:
        float: 优化后电路的面积延迟积相对于基准电路的改善比例
        """
        next_state = "aig_optimized.aig"
        next_bench = "aig_optimized.bench"
    
        # 步骤 1: 使用 ABC 工具评估初始 AIG 文件
        abcRunCmd = f"./yosys-abc -c \"read {aig_file}; read_lib {lib_file}; map; topo; stime\" > {log_file}"
        os.system(abcRunCmd)
    
        with open(log_file) as f:
            area_information = re.findall(r'[a-zA-Z0-9.]+', f.readlines()[-1])
            baseline = float(area_information[-9]) * float(area_information[-4])
    
        # 步骤 2: 使用 RESYN2 优化 AIG 文件
        RESYN2_CMD = "balance; rewrite; refactor; balance; rewrite; rewrite -z; balance; refactor -z; rewrite -z; balance;"
        abcRunCmd = f"./yosys-abc -c \"read {aig_file}; {RESYN2_CMD} read_lib {lib_file}; write {next_state}; write_bench -l {next_bench}; map; topo; stime\" > {log_file}"
        os.system(abcRunCmd)
    
        with open(log_file) as f:
            area_information = re.findall(r'[a-zA-Z0-9.]+', f.readlines()[-1])
        optimized = float(area_information[-9]) * float(area_information[-4])
    
        # 计算优化后电路的面积延迟积相对于基准电路的改善比例
        eval = 1 - optimized / baseline
        return eval
    
    data_dir = '/home/genshinmaster/project/project_data'
    
    for filename in os.listdir(data_dir):
        if filename.endswith('.pkl'):
            file_path = os.path.join(data_dir, filename)
            with open(file_path, 'rb') as f:
                data = pickle.load(f)
                if data:
                    logicSynthesis(data['input'][-1])
    # 调用函数进行逻辑合成
               
\end{lstlisting} 


\subsection{获取AIG的data表示}
我们按照示例代码实现try3.py,用于读取AIG文件对应的data字典,
\begin{lstlisting}
    import abc_py
    import numpy as np
    import torch
    
    def read_aig(state):
        """
        读取AIG文件并返回表示AIG的data字典
        """
        _abc = abc_py.AbcInterface()
        _abc.start()
        _abc.read(state)
    
        data = {}
        numNodes = _abc.numNodes()
        data['node_type'] = np.zeros(numNodes, dtype=int)
        data['num_inverted_predecessors'] = np.zeros(numNodes, dtype=int)
    
        edge_src_index = []
        edge_target_index = []
    
        for nodeIdx in range(numNodes):
            aigNode = _abc.aigNode(nodeIdx)
            nodeType = aigNode.nodeType()
            data['num_inverted_predecessors'][nodeIdx] = 0
    
            if nodeType == 0 or nodeType == 2:
                data['node_type'][nodeIdx] = 0
            elif nodeType == 1:
                data['node_type'][nodeIdx] = 1
            else:
                data['node_type'][nodeIdx] = 2
    
            if nodeType == 4:
                data['num_inverted_predecessors'][nodeIdx] = 1
            elif nodeType == 5:
                data['num_inverted_predecessors'][nodeIdx] = 2
    
            if (aigNode.hasFanin0()):
                fanin = aigNode.fanin0()
                edge_src_index.append(nodeIdx)
                edge_target_index.append(fanin)
            if (aigNode.hasFanin1()):
                fanin = aigNode.fanin1()
                edge_src_index.append(nodeIdx)
                edge_target_index.append(fanin)
    
        data['edge_index'] = torch.tensor([edge_src_index, edge_target_index], dtype=torch.long)
        data['node_type'] = torch.tensor(data['node_type'])
        data['num_inverted_predecessors'] = torch.tensor(data['num_inverted_predecessors'])
        data['nodes'] = numNodes
    
        return data
        
        
        
    # 读取AIG文件
    aig_file_path = "/home/genshinmaster/yosys/alu2_0130622.aig"
    data = read_aig(aig_file_path)
    
    # 查看data字典的内容
    print(data.keys())
    # 输出: dict_keys(['node_type', 'num_inverted_predecessors', 'edge_index', 'nodes'])
    
    # 获取节点类型信息
    print(data['node_type'])
    # 输出: tensor([0, 1, 0, ..., 2, 0, 1])
    
    # 获取反向前驱节点个数信息 
    print(data['num_inverted_predecessors'])
    # 输出: tensor([0, 1, 0, ..., 2, 0, 1])
    
    # 获取边信息
    print(data['edge_index'])
    # 输出: tensor([[0, 0, 1, 1, ...], 
    #               [1, 2, 2, 3, ...]])
    
    # 获取节点数量
    print(data['nodes'])
    # 输出: 1234
\end{lstlisting}
输出如下图,这里展示了data字典的内容,节点类型,反向前驱节点个数,边index,节点数量等信息。
\begin{figure}[H]
    \centering
    \includegraphics[width=\textwidth]{try3.png}
    
    \caption{}
    \label{fig:try3}
\end{figure}\par

\subsection{最终实现}
有了AIG获取和转换成data字典的代码,我们可以给出如下模型,进行训练。

我们读取300个\.pkl文件的数据并将其划分为测试集和训练集,
\begin{lstlisting}{language=python}
    import os
    import pickle
    import torch
    import torch.nn as nn
    import torch.nn.functional as F
    import numpy as np
    from torch_geometric.nn import GCNConv
    from torch_geometric.utils import add_self_loops
    import abc_py
    from sklearn.model_selection import train_test_split
    
    def logicSynthesis(state):
        circuitName, actions = state.split('_')
        circuitPath = f'/home/genshinmaster/project/InitialAIG/train/{circuitName}.aig'
        libFile = '/home/genshinmaster/project/lib/7nm/7nm.lib'
        logFile = f'{circuitName}.log'
        nextState = f'/home/genshinmaster/yosys/AIGFiles/{state}.aig'
    
        synthesisOpToPosDic = {
            0: "refactor",
            1: "refactor -z",
            2: "rewrite",
            3: "rewrite -z",
            4: "resub",
            5: "resub -z",
            6: "balance"
        }
    
        action_cmd = ''
        for digit in actions:
            if digit.isdigit():
                operation = int(digit)
                if operation in synthesisOpToPosDic:
                    action_cmd += f"{synthesisOpToPosDic[operation]}; "
    
        abcRunCmd = f"./yosys-abc -c \"read {circuitPath}; {action_cmd}; read_lib {libFile}; write {nextState}; print_stats\" > {logFile}"
        os.system(abcRunCmd)
    
        return nextState
    
    def read_aig(state):
        _abc = abc_py.AbcInterface()
        _abc.start()
        _abc.read(state)
    
        data = {}
        numNodes = _abc.numNodes()
        data['node_type'] = np.zeros(numNodes, dtype=int)
        data['num_inverted_predecessors'] = np.zeros(numNodes, dtype=int)
    
        edge_src_index = []
        edge_target_index = []
    
        for nodeIdx in range(numNodes):
            aigNode = _abc.aigNode(nodeIdx)
            nodeType = aigNode.nodeType()
            data['num_inverted_predecessors'][nodeIdx] = 0
    
            if nodeType == 0 or nodeType == 2:
                data['node_type'][nodeIdx] = 0
            elif nodeType == 1:
                data['node_type'][nodeIdx] = 1
            else:
                data['node_type'][nodeIdx] = 2
    
            if nodeType == 4:
                data['num_inverted_predecessors'][nodeIdx] = 1
            elif nodeType == 5:
                data['num_inverted_predecessors'][nodeIdx] = 2
    
            if aigNode.hasFanin0():
                fanin = aigNode.fanin0()
                edge_src_index.append(nodeIdx)
                edge_target_index.append(fanin)
            if aigNode.hasFanin1():
                fanin = aigNode.fanin1()
                edge_src_index.append(nodeIdx)
                edge_target_index.append(fanin)
    
        data['edge_index'] = torch.tensor([edge_src_index, edge_target_index], dtype=torch.long)
        data['node_type'] = torch.tensor(data['node_type'])
        data['num_inverted_predecessors'] = torch.tensor(data['num_inverted_predecessors'])
        data['nodes'] = numNodes
    
        return data
    
    class GNNModel(nn.Module):
        def __init__(self, in_channels, hidden_channels, out_channels):
            super(GNNModel, self).__init__()
            self.conv1 = GCNConv(in_channels, hidden_channels)
            self.conv2 = GCNConv(hidden_channels, hidden_channels)
            self.conv3 = GCNConv(hidden_channels, out_channels)
            self.activation = nn.ReLU()
    
        def forward(self, x, edge_index):
            edge_index, _ = add_self_loops(edge_index, num_nodes=x.size(0))
            
            x = self.conv1(x, edge_index)
            x = self.activation(x)
            x = self.conv2(x, edge_index)
            x = self.activation(x)
            x = self.conv3(x, edge_index)
            
            x = torch.mean(x, dim=0)  # Compute the mean of all node outputs
            return x.view(1)  # Ensure the output shape is [1]
    
    def prepare_input_data(input_data):
        data, score = input_data
        x = []
        if 'node_type' in data:
            x.append(data['node_type'].unsqueeze(1).float())
        if 'num_inverted_predecessors' in data:
            x.append(data['num_inverted_predecessors'].unsqueeze(1).float())
        x = torch.cat(x, dim=1) if x else torch.zeros((data['nodes'], 0), dtype=torch.float)
        edge_index = data['edge_index'].long()
        score = torch.tensor(score, dtype=torch.float).view(1)  # Ensure the score shape is [1]
    
        # 检查 `nan` 值
        assert not torch.isnan(x).any(), f"x contains nan: {x}"
        assert not torch.isnan(edge_index).any(), f"edge_index contains nan: {edge_index}"
        assert not torch.isnan(score).any(), f"score contains nan: {score}"
        
        return x, edge_index, score
    
    # 检查文件是否存在的辅助函数
    def check_file_exists(file_path):
        if not os.path.exists(file_path):
            print(f"{file_path} 文件不存在")
            return False
        return True
    
    # 加载数据集
    def load_dataset(file_prefix, range_start, range_end):
        dataset = []
        for i in range(range_start, range_end + 1):
            pkl_file = f"{file_prefix}_{i}.pkl"
            pkl_path = os.path.join('/home/genshinmaster/project/project_data', pkl_file)
            if check_file_exists(pkl_path):
                with open(pkl_path, 'rb') as file:
                    data = pickle.load(file)
                    input_list = data['input']
                    score_list = data['target']
                    for j in range(1, len(input_list)):
                        state = input_list[j]
                        nextstate = logicSynthesis(state)
                        data = read_aig(nextstate)
                        input_data = (data, score_list[j])
                        dataset.append(input_data)
        return dataset
    
    # 加载 adder, alu2 和 apex3 数据集
    print("begin load data1")
    train_dataset = load_dataset('adder', 0, 100)
    print("begin load data2")
    train_dataset += load_dataset('alu2', 0, 100)
    print("begin load data3")
    train_dataset += load_dataset('apex3', 0, 100)
    # 使用train_test_split划分训练集和测试集
    train_dataset, test_dataset = train_test_split(train_dataset, test_size=0.2, random_state=42)
    
    # 确保训练集和测试集不为空
    if not train_dataset:
        print("训练集为空,请检查数据文件。")
    if not test_dataset:
        print("测试集为空,请检查数据文件。")
    
    num_features = 2  # node_type and num_inverted_predecessors
    model = GNNModel(in_channels=num_features, hidden_channels=64, out_channels=1)
    optimizer = torch.optim.Adam(model.parameters(), lr=0.001)
    
    # 训练模型
    for epoch in range(100):
        model.train()
        for input_data in train_dataset:
            optimizer.zero_grad()
            x, edge_index, score = prepare_input_data(input_data)
            output = model(x, edge_index)
            loss = F.mse_loss(output, score)  # Ensure output and score are both [1]
            print(f"训练损失: {loss.item()}")  # 打印训练损失以便调试
            loss.backward()
            optimizer.step()
    
    # 评估模型
    print("finish train")
    model.eval()
    with torch.no_grad():
        test_predictions = []
        test_targets = []
    
        for input_data in test_dataset:
            x, edge_index, score = prepare_input_data(input_data)
            
            # 打印输入数据以便调试
            print(f"x: {x}")
            print(f"edge_index: {edge_index}")
            print(f"score: {score}")
            
            output = model(x, edge_index)
            
            # 确保输出没有 `nan` 值
            assert not torch.isnan(output).any(), f"output contains nan: {output}"
            
            # 打印模型输出和目标值
            print(f"模型输出: {output.item()}, 目标值: {score.item()}")
            
            test_predictions.append(output.item())
            test_targets.append(score.item())
    
        test_predictions = torch.tensor(test_predictions)
        test_targets = torch.tensor(test_targets)
        assert not torch.isnan(test_predictions).any(), f"test_predictions contains nan: {test_predictions}"
        assert not torch.isnan(test_targets).any(), f"test_targets contains nan: {test_targets}"
    
        mse = F.mse_loss(test_predictions, test_targets)
        print(f"Test MSE: {mse.item()}")
    
\end{lstlisting}
输出结果如下图,TEST MSE接近于0,说明模型的效果很好
\begin{figure}[H]
    \centering
    \includegraphics[width=\textwidth]{datapre.png}
    
    \caption{}
    \label{fig:datapre}
\end{figure}\par



\section{任务2:逻辑综合决策}
\subsection{FutureRewardModel}
由已知,我们需要训练一个model用于处理futureReward,代码如下所示
\begin{lstlisting}{language=python}
    import os
    import pickle
    import torch
    import torch.nn as nn
    import torch.nn.functional as F
    import numpy as np
    from torch.utils.data import Dataset, DataLoader
    from sklearn.model_selection import train_test_split
    import itertools
    
    # 数据集加载函数
    def load_dataset(file_prefix, range_start, range_end):
        dataset = []
        for i in range(range_start, range_end + 1):
            pkl_file = f"{file_prefix}_{i}.pkl"
            pkl_path = os.path.join('/home/genshinmaster/project/project_data', pkl_file)
            if os.path.exists(pkl_path):
                with open(pkl_path, 'rb') as file:
                    data = pickle.load(file)
                    input_list = data['input']
                    target_list = data['target']
                    for j in range(len(input_list)):
                        state = input_list[j]
                        expected_future_reward = target_list[j]
                        dataset.append((state, expected_future_reward))
        return dataset
    
    # 加载 adder, alu2 数据集
    adder_dataset = load_dataset('adder', 0, 100)
    alu2_dataset = load_dataset('alu2', 0, 100)
    
    # 将数据集合并
    full_dataset = adder_dataset + alu2_dataset
    
    # 划分训练集和验证集
    train_dataset, val_dataset = train_test_split(full_dataset, test_size=0.2, random_state=42)
    
    
    class AIGDataset(Dataset):
        def __init__(self, data):
            self.data = data
    
        def __len__(self):
            return len(self.data)
    
        def __getitem__(self, idx):
            state, future_reward = self.data[idx]
            features = process_state(state)
            return features.float(), torch.tensor(future_reward).float()
    
    
    
    def process_state(state):
        char_set = 'abcdefghijklmnopqrstuvwxyz0123456789_'  # 字符集合,包括字母、数字和下划线
        num_chars = len(char_set)
        max_length = 10  # 假设状态的最大长度为10
    
        features = torch.zeros(max_length, num_chars)
    
        for i, char in enumerate(state[:max_length]):
            if char in char_set:
                features[i, char_set.index(char)] = 1
    
        features_flat = features.view(-1)
    
        return features_flat
    
    
    class FutureRewardModel(nn.Module):
        def __init__(self, input_dim, hidden_size, output_size):
            super(FutureRewardModel, self).__init__()
            self.fc1 = nn.Linear(input_dim, hidden_size)
            self.fc2 = nn.Linear(hidden_size, output_size)  # 输出预期未来奖励值
            self.activation = nn.ReLU()
    
        def forward(self, x):
            x = self.activation(self.fc1(x))
            x = self.fc2(x)
            return x
    
    
    # 参数设置
    input_dim = 370  # 根据状态的特征向量长度调整
    hidden_dim = 64
    batch_size = 32
    num_epochs = 50
    learning_rate = 0.001
    
    # 准备数据集和数据加载器
    train_dataset = AIGDataset(train_dataset)
    val_dataset = AIGDataset(val_dataset)
    train_loader = DataLoader(train_dataset, batch_size=batch_size, shuffle=True)
    val_loader = DataLoader(val_dataset, batch_size=batch_size)
    
    # 初始化模型和优化器
    model = FutureRewardModel(input_dim, hidden_dim, 1)
    optimizer = torch.optim.Adam(model.parameters(), lr=learning_rate)
    criterion = nn.MSELoss()
    
    # 训练循环
    for epoch in range(num_epochs):
        model.train()
        train_loss = 0.0
        for inputs, targets in train_loader:
            optimizer.zero_grad()
            outputs = model(inputs)
            loss = criterion(outputs, targets.unsqueeze(1))  # 使用 MSE 损失
            loss.backward()
            optimizer.step()
            train_loss += loss.item() * inputs.size(0)
        
        # 计算验证集损失
        model.eval()
        val_loss = 0.0
        with torch.no_grad():
            for inputs, targets in val_loader:
                outputs = model(inputs)
                loss = criterion(outputs, targets.unsqueeze(1))
                val_loss += loss.item() * inputs.size(0)
        
        train_loss /= len(train_loader.dataset)
        val_loss /= len(val_loader.dataset)
        
        print(f"Epoch [{epoch+1}/{num_epochs}], Train Loss: {train_loss:.4f}, Val Loss: {val_loss:.4f}")
    
    # 保存训练好的模型
    torch.save(model.state_dict(), '/home/genshinmaster/project/future_reward_model.pth')
    
\end{lstlisting}
代码运行结果如下,可以看到loss逐渐减小,说明模型训练较为合理
\begin{figure}[H]
    \centering
    \includegraphics[width=\textwidth]{futureReward.png}
    
    \caption{}
    \label{fig:futureReward}
\end{figure}\par
\subsection{搜索算法实现}
基于训练好的两个模型,对于给定名称的alu2这样的aig,我们可以遍历可能的形如alu2\_4213231之类的state,
可以使用FutureRewardModel来给出预测评分,最终取最优的即可
代码如下所示:
\begin{lstlisting}{language=python}
    import os
    import pickle
    import torch
    import torch.nn as nn
    import torch.nn.functional as F
    import numpy as np
    from torch.utils.data import Dataset, DataLoader
    from sklearn.model_selection import train_test_split
    import itertools
    
    
    
    class AIGDataset(Dataset):
        def __init__(self, data):
            self.data = data
    
        def __len__(self):
            return len(self.data)
    
        def __getitem__(self, idx):
            state, future_reward = self.data[idx]
            features = process_state(state)
            return features.float(), torch.tensor(future_reward).float()
    
    
    
    def process_state(state):
        char_set = 'abcdefghijklmnopqrstuvwxyz0123456789_'  # 字符集合,包括字母、数字和下划线
        num_chars = len(char_set)
        max_length = 10  # 假设状态的最大长度为10
    
        features = torch.zeros(max_length, num_chars)
    
        for i, char in enumerate(state[:max_length]):
            if char in char_set:
                features[i, char_set.index(char)] = 1
    
        features_flat = features.view(-1)
    
        return features_flat
    
    
    class FutureRewardModel(nn.Module):
        def __init__(self, input_dim, hidden_size, output_size):
            super(FutureRewardModel, self).__init__()
            self.fc1 = nn.Linear(input_dim, hidden_size)
            self.fc2 = nn.Linear(hidden_size, output_size)  # 输出预期未来奖励值
            self.activation = nn.ReLU()
    
        def forward(self, x):
            x = self.activation(self.fc1(x))
            x = self.fc2(x)
            return x
    
    
    # 参数设置
    input_dim = 370  # 根据状态的特征向量长度调整
    hidden_dim = 64
    batch_size = 32
    num_epochs = 50
    learning_rate = 0.001
    
    
    
    
    # 初始化模型和优化器
    model = FutureRewardModel(input_dim, hidden_dim, 1)
    optimizer = torch.optim.Adam(model.parameters(), lr=learning_rate)
    criterion = nn.MSELoss()
    
    # 加载已训练好的模型参数
    model_path = '/home/genshinmaster/project/future_reward_model.pth'
    model.load_state_dict(torch.load(model_path))
    
    
    # 定义动作空间生成函数
    def generate_actions(prefix='alu2_', max_digits=10, digit_range=range(7), max_actions=None):
        actions = []
        num_actions = 0  # 记录已生成的动作数量
        print("generate_actions")
        
        for num_digits in range(1, max_digits + 1):
            if max_actions is not None and num_actions >= max_actions:
                break
            
            for combo in itertools.product(digit_range, repeat=num_digits):
                if max_actions is not None and num_actions >= max_actions:
                    break
                
                action = prefix + ''.join(map(str, combo))
                actions.append(action)
                num_actions += 1
        
        return actions
    
    # 生成动作空间,限制最多生成max_actions 个state
    max_actions = 100
    actions = generate_actions(prefix='alu2_', max_digits=10, digit_range=range(7), max_actions=max_actions)
    
    
    
    
    # 使用贪婪搜索来选择最优动作
    def greedy_search(actions, model):
        best_action = None
        best_score = float('-inf')
        
        for action in actions:
            # 将动作转换为特征表示
            action_features = process_state(action)
            
            # 将特征表示转换为模型期望的输入形式(添加批次维度)
            action_features = action_features.unsqueeze(0)
            
            # 使用评分模型评估每个动作的评分
            action_score = model(action_features).item()
            
            # 更新最优动作和评分
            if action_score > best_score:
                best_action = action
                best_score = action_score
        
        return best_action, best_score
    
    best_action, best_score = greedy_search(actions, model)
    print(f"Best Action: {best_action}, Best Score: {best_score:.4f}")
\end{lstlisting}


我们分别将max\_actions设置为100,1000,10000,100000,得到的结果如下,可以看出解空间越大,对应的最终结果就越好。


\begin{figure}[H]
    \centering
    \includegraphics[width=\textwidth]{search1.png}
    
    \caption{}
    \label{fig:search}
\end{figure}\par








\end{document}

